\documentclass[12pt]{article}
\usepackage[T2A]{fontenc}
\usepackage[utf8]{inputenc}        % Кодировка входного документа;
                                    % при необходимости, вместо cp1251
                                    % можно указать cp866 (Alt-кодировка
                                    % DOS) или koi8-r.

\usepackage[english,russian]{babel} % Включение русификации, русских и
                                    % английских стилей и переносов

\usepackage{amsmath,amsfonts,amsthm,amssymb,amsbsy,amstext,amscd,amsxtra,multicol}
\usepackage{verbatim}
\usepackage{tikz} %Рисование автоматов
\usetikzlibrary{automata,positioning}
\usepackage{multicol} %Несколько колонок
\usepackage{graphicx}
\usepackage[colorlinks,urlcolor=blue]{hyperref}
\usepackage[stable]{footmisc}

\usepackage{amsmath,amssymb}
\DeclareMathOperator{\E}{\mathbb{E}}


%греческие буквы
\let\epsilon\varepsilon
\let\es\emptyset
\let\eps\varepsilon
\let\al\alpha
\let\sg\sigma
\let\ga\gamma
\let\ph\varphi
\let\om\omega
\let\ld\lambda
\let\Ld\Lambda
\let\vk\varkappa
\let\Om\Omega
\def\abstractname{}

\let\yield\Rightarrow

\def\R{{\cal R}}
\def\A{{\cal A}}
\def\B{{\cal B}}
\def\C{{\cal C}}
\def\D{{\cal D}}

\def\w{{\mathbf{w} }}
\def\y{{\mathbf{y}}}
\def\h{{\mathbf h }}

%классы сложности
\def\REG{{\mathsf{REG}}}
\def\CFL{{\mathsf{CFL}}}



\newcounter{problem}
\newcounter{uproblem}
\newcounter{subproblem}
\newcounter{prvar}




\def\pr{\medskip\noindent\stepcounter{problem}{\bf \theproblem .  }\setcounter{subproblem}{0} }
\def\prstar{\medskip\noindent\stepcounter{problem}{\bf $\mathbf{\theproblem}^*$\negthickspace.  }\setcounter{subproblem}{0} }
\def\prpfrom[#1]{\medskip\noindent\stepcounter{problem}{\bf Задача \theproblem~(№#1 из задания).  }\setcounter{subproblem}{0} }
\def\prp{\medskip\noindent\stepcounter{problem}{\bf Задача \theproblem .  }\setcounter{subproblem}{0} }


\def\prsub{\medskip\noindent\stepcounter{subproblem}{\sf \thesubproblem .} }




\title{\Large Первое теоретическое задание}
\author{Фельдман Даниил}
\date{}






\begin{document}

\maketitle

Докажем утверждение о том, что максимизация вариационной нижней оценки эквивалентна минимизации KL-дивергенции между распределением $q(\w)\in Q$ и апостериорным распределением параметров $p(\w |\h ,X,\y)$:
$$\hat{q} = \arg\max\limits_{q\in Q}\int\limits_{\w} q(\w)\log \frac{p(\y,\w|X,\h)}{q(\w)} d\w \Leftrightarrow \hat{q} = \arg\min\limits_{q\in Q} D_{KL} (q(\w)||p(\w|\y,X,\h))$$ 

$$D_{KL}(q(\w)||p(\w|\y,X,\h)) = \int\limits_{\w}q(\w)\log\frac{q(\w)}{p(\w|\y,X,\h)} = $$
$$= -\int\limits_{\w}q(\w)\log\frac{p(\w|\y,X,\h)}{q(\w)} =$$
$$= -(\int\limits_{\w}q(\w)\log\frac{p(\y,\w|X,\h)}{q(\w)}d\w - \int\limits_{w}q(\w)\log p(\y|X,\h)d\w) = $$
$$= -(\int\limits_{\w}q(\w)\log\frac{p(\y,\w|X,\h)}{q(\w)}d\w - \log p(\y|X,\h)\int\limits_{w}q(\w)d\w) = $$
$$= -\int\limits_{\w}q(\w)\log\frac{p(\y,\w|X,\h)}{q(\w)}d\w + \log p(\y|X,\h)$$

Мы получили два слагаемых, второе из которых не зависит от $\w$. Поэтому минимизация KL-дивергенции эквивалентна максимизации первого слагаемого без знака минус, что и является вариационной нижней оценкой.

\end{document}
