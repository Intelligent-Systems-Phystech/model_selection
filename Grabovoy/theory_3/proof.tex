\documentclass[12pt, twoside]{article}
\usepackage[utf8]{inputenc}
\usepackage[english,russian]{babel}

\usepackage{amsthm}
\usepackage{a4wide}
\usepackage{graphicx}
\usepackage{caption}
\usepackage{amssymb}
\usepackage{amsmath}
\usepackage{mathrsfs}
\usepackage{euscript}
\usepackage{graphicx}
\usepackage{subfig}
\usepackage{caption}
\usepackage{color}
\usepackage{bm}
\usepackage{tabularx}
\usepackage{adjustbox}

\newtheorem{theorem}{Теорема}



\begin{document}
\begin{theorem}
Пусть
\begin{enumerate}
\item Заданы компактные множества $\mathbf{U}_{\mathbf{h}} \subset \mathbb{H}, \mathbf{U}_{\bm{\theta}_{\mathbf{w}}} \subset \bm{\theta}_{\mathbf{w}}, \mathbf{U}_{\bm{\theta}_{\bm{\Gamma}}} \subset \bm{\theta}_{\bm{\Gamma}}$.

\item Вариационное распределение $q_{\mathbf{w}}\left(\mathbf{w}|\bm{\Gamma}, \bm{\theta}_{\mathbf{w}}\right)$  является абсолютно непрерывным и унимодальным на  $U_{\bm{\theta}}$.
Его мода и матожидание совпадают:

\[
  \textbf{mode}~q_{\mathbf{w}}\left(\mathbf{w}|\bm{\Gamma}, \bm{\theta}_{\mathbf{w}}\right)= \mathsf{E}_{q_{\mathbf{w}}\left(\mathbf{w}|\bm{\Gamma}, \bm{\theta}_{\mathbf{w}}\right)} \mathbf{w}.
\]




\item Априорное распределение $p\left(\mathbf{w}|\bm{\Gamma}, \mathbf{h}, \bm{\lambda}\right)$ является абсолютно непрерывным и унимодальным на  $U_\mathbf{h}$. Его мода и матожидание совпадают и не зависят от гиперпараметров $\mathbf{h}$  на $\mathbf{U}_{\mathbf{h}}$ и структуры $\Gamma$ на $\mathbf{U}_{\bm{\theta}_{\bm{\Gamma}}}$:
$
\mathsf{E}_{p\left(\mathbf{w}|\bm{\Gamma}, \mathbf{h}, \bm{\lambda}\right)}~\mathbf{w} = \textbf{mode}~p\left(\mathbf{w}|\bm{\Gamma}_1, \mathbf{h}_1, \bm{\lambda}\right)=\textbf{mode}~p\left(\mathbf{w}|\bm{\Gamma}_1, \mathbf{h}_2, \bm{\lambda}\right)=\mathbf{m}
$
\text{ для любых }$~\mathbf{h}_1,\mathbf{h}_2 \in \mathbf{U}_{\mathbf{h}}, \Gamma_1,\Gamma_2 \in \mathbf{U}_{\bm{\Gamma}}$.


\item Параметры модели $\mathbf{w}$ имеют конечные вторые моменты по маргинальным распределениям:
$
   \int_{\Gamma}q_{\bm{\Gamma}}\left(\bm{\Gamma}|\bm{\theta}_{\bm{\Gamma}}\right)q_{\mathbf{w}}\left(\mathbf{w}|\bm{\Gamma}, \bm{\theta}_{\mathbf{w}}\right)_{\mathbf{w}} d\Gamma, \quad \int_{\Gamma}q_{\bm{\Gamma}}\left(\bm{\Gamma}|\bm{\theta}_{\bm{\Gamma}}\right)p\left(\mathbf{w}|\bm{\Gamma}, \mathbf{h}, \bm{\lambda}\right) d\Gamma
$
при любых $\bm{\theta}_{\mathbf{w}}\in \mathbf{U}_{\bm{\theta}_{\mathbf{w}}}, \bm{\theta}_{\bm{\Gamma}} \in \mathbf{U}_{\bm{\theta}_{\bm{\Gamma}}}, \mathbf{h} \in \mathbf{U}_{\mathbf{h}}.$


\item Вариационное распределение $q_{\mathbf{w}}\left(\mathbf{w}|\bm{\Gamma}, \bm{\theta}_{\mathbf{w}}\right)$ является липшицевым по $\mathbf{w}$ с параметров~$L$.

\item Значение $q_{\mathbf{w}}\left(\mathbf{w}|\bm{\Gamma}, \bm{\theta}_{\mathbf{w}}\right)$ не равно нулю при любых $\bm{\theta} \in \mathbf{U}_{\bm{\theta}}, \Gamma \in \mathbb{G}$.
\item Точная нижняя грань $q_{\mathbf{w}}\left(\mathbf{m}|\bm{\Gamma}, \bm{\theta}_{\mathbf{w}}\right)$ не равна нулю при $\bm{\theta}_{\mathbf{w}}\in \mathbf{U}_{\bm{\theta}_{\mathbf{w}}}$ и $\Gamma \in \mathbb{G}$:
\[
    \inf_{\Gamma \in \mathbb{G}, \bm{\theta}_{\mathbf{w}}\in \mathbf{U}_{\bm{\theta}_{\mathbf{w}}}} q_{\mathbf{w}}\left(\mathbf{m}|\bm{\Gamma}, \bm{\theta}_{\mathbf{w}}\right) > 0.
\]
\end{enumerate}
Тогда 
\[
   \left|\mathsf{E}_{q_{\bm{\Gamma}}\left(\bm{\Gamma}|\bm{\theta}_{\bm{\Gamma}}\right)} {\rho}(\mathbf{w}|\Gamma, \bm{\theta}_{\mathbf{w}}, \mathbf{h}, \bm{\lambda})^{-1} - 1\right| 
\leq \text{Const } \iint_{\Gamma,\mathbf{w}} |\mathbf{w}| \cdot |q_{\mathbf{w}}\left(\mathbf{w}|\bm{\Gamma}, \bm{\theta}_{\mathbf{w}}\right) - p\left(\mathbf{w}|\bm{\Gamma}, \mathbf{h}, \bm{\lambda}\right)|{q_{\bm{\Gamma}}\left(\bm{\Gamma}|\bm{\theta}_{\bm{\Gamma}}\right)} d\mathbf{w} d\Gamma.
\]
\end{theorem}

\footnotetext[1]{Занесли модуль под знак интеграла}
\footnotetext[2]{Условие Лившица из 5}
\footnotetext[3]{Из 7го условия теоремы}
\footnotetext[4]{Из 2 и 3 условия}
\footnotetext[5]{Перепишем математическое ожидание через интеграл и внесем модуль под знак интеграла}

\begin{proof}
\begin{equation*}
\begin{aligned}
&\left|\mathsf{E}_{q_{\bm{\Gamma}}\left(\bm{\Gamma}|\bm{\theta}_{\bm{\theta}}\right)}\rho\left(\mathbf{w}|\bm{\Gamma}, \bm{\theta}_{\mathbf{w}}, \mathbf{h}, \bm{\lambda}\right)^{-1} - 1\right| = 
\left|\int_{\bm{\Gamma}}\left(\frac{q_{\mathbf{w}\left(\text{mode}~q_{\mathbf{w}}\left(\mathbf{w}|\bm{\Gamma}, \bm{\theta}_{\mathbf{w}}\right)|\bm{\Gamma}, \bm{\theta}_{\mathbf{w}}\right)}}{q_{\mathbf{w}\left(\text{mode}~p_{\mathbf{w}}\left(\mathbf{w}|\bm{\Gamma}, \mathbf{h}, \bm{\lambda}\right)|\bm{\Gamma}, \bm{\theta}_{\mathbf{w}}\right)}}\right)q_{\bm{\Gamma}}\left(\bm{\Gamma}|\bm{\theta}_{\bm{\Gamma}}\right)d\bm{\Gamma} - 1\right| = \\
&= \left|\int_{\bm{\Gamma}}\left(\frac{q_{\mathbf{w}\left(\text{mode}~q_{\mathbf{w}}\left(\mathbf{w}|\bm{\Gamma}, \bm{\theta}_{\mathbf{w}}\right)|\bm{\Gamma}, \bm{\theta}_{\mathbf{w}}\right)} - q_{\mathbf{w}\left(\text{mode}~p_{\mathbf{w}}\left(\mathbf{w}|\bm{\Gamma}, \mathbf{h}, \bm{\lambda}\right)|\bm{\Gamma}, \bm{\theta}_{\mathbf{w}}\right)}}{q_{\mathbf{w}\left(\text{mode}~p_{\mathbf{w}}\left(\mathbf{w}|\bm{\Gamma}, \mathbf{h}, \bm{\lambda}\right)|\bm{\Gamma}, \bm{\theta}_{\mathbf{w}}\right)}}\right)q_{\bm{\Gamma}}\left(\bm{\Gamma}|\bm{\theta}_{\bm{\Gamma}}\right)d\bm{\Gamma}\right| \leq\footnote{1} \\
&\leq \int_{\bm{\Gamma}}\left|\left(\frac{q_{\mathbf{w}\left(\text{mode}~q_{\mathbf{w}}\left(\mathbf{w}|\bm{\Gamma}, \bm{\theta}_{\mathbf{w}}\right)|\bm{\Gamma}, \bm{\theta}_{\mathbf{w}}\right)} - q_{\mathbf{w}\left(\text{mode}~p_{\mathbf{w}}\left(\mathbf{w}|\bm{\Gamma}, \mathbf{h}, \bm{\lambda}\right)|\bm{\Gamma}, \bm{\theta}_{\mathbf{w}}\right)}}{q_{\mathbf{w}\left(\text{mode}~p_{\mathbf{w}}\left(\mathbf{w}|\bm{\Gamma}, \mathbf{h}, \bm{\lambda}\right)|\bm{\Gamma}, \bm{\theta}_{\mathbf{w}}\right)}}\right)\right|q_{\bm{\Gamma}}\left(\bm{\Gamma}|\bm{\theta}_{\bm{\Gamma}}\right)d\bm{\Gamma} \leq\footnote{2}\\
&\leq  L\int_{\bm{\Gamma}}\left(\frac{\left|\text{mode}~q_{\mathbf{w}}\left(\mathbf{w}|\bm{\Gamma}, \bm{\theta}_{\mathbf{w}}\right) - \text{mode}~p_{\mathbf{w}}\left(\mathbf{w}|\bm{\Gamma}, \mathbf{h}, \bm{\lambda}\right)\right|}{q_{\mathbf{w}\left(\text{mode}~p_{\mathbf{w}}\left(\mathbf{w}|\bm{\Gamma}, \mathbf{h}, \bm{\lambda}\right)|\bm{\Gamma}, \bm{\theta}_{\mathbf{w}}\right)}}\right)q_{\bm{\Gamma}}\left(\bm{\Gamma}|\bm{\theta}_{\bm{\Gamma}}\right)d\bm{\Gamma} \leq\footnote{3}\\
&\leq Const\int_{\bm{\Gamma}}\left|\text{mode}~q_{\mathbf{w}}\left(\mathbf{w}|\bm{\Gamma}, \bm{\theta}_{\mathbf{w}}\right) - \text{mode}~p_{\mathbf{w}}\left(\mathbf{w}|\bm{\Gamma}, \mathbf{h}, \bm{\lambda}\right)\right|q_{\bm{\Gamma}}\left(\bm{\Gamma}|\bm{\theta}_{\bm{\Gamma}}\right)d\bm{\Gamma} =\footnote{4} \\
& = Const\int_{\bm{\Gamma}}\left|\mathsf{E}_{q_{\mathbf{w}}\left(\mathbf{w}|\bm{\Gamma}, \bm{\theta}_{\mathbf{w}}\right)}\mathbf{w} - \mathsf{E}_{p_{\mathbf{w}}\left(\mathbf{w}|\bm{\Gamma}, \mathbf{h}, \bm{\lambda}\right)}\mathbf{w}\right|q_{\bm{\Gamma}}\left(\bm{\Gamma}|\bm{\theta}_{\bm{\Gamma}}\right)d\bm{\Gamma} \leq\footnote{5}\\
&\leq Const\int_{\bm{\Gamma}}\int_{\mathbf{w}}\left|\mathbf{w}\right|\left|{q_{\mathbf{w}}\left(\mathbf{w}|\bm{\Gamma}, \bm{\theta}_{\mathbf{w}}\right)} - {p_{\mathbf{w}}\left(\mathbf{w}|\bm{\Gamma}, \mathbf{h}, \bm{\lambda}\right)}\right|q_{\bm{\Gamma}}\left(\bm{\Gamma}|\bm{\theta}_{\bm{\Gamma}}\right)d\mathbf{w}d\bm{\Gamma}
\end{aligned}
\end{equation*}
\end{proof}

\end{document} 

