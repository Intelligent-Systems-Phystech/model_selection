\documentclass[12pt, twoside]{article}
\usepackage[utf8]{inputenc}
\usepackage[english,russian]{babel}

\usepackage{amsthm}
\usepackage{a4wide}
\usepackage{graphicx}
\usepackage{caption}
\usepackage{amssymb}
\usepackage{amsmath}
\usepackage{mathrsfs}
\usepackage{euscript}
\usepackage{graphicx}
\usepackage{subfig}
\usepackage{caption}
\usepackage{color}
\usepackage{bm}
\usepackage{tabularx}
\usepackage{adjustbox}

\newtheorem{theorem}{Теорема}



\begin{document}
\begin{theorem}
Пусть $\mathcal{L}$ --- дифференцируемая функция, такая что все стационарные точки $\mathcal{L}$ являются локальными минимумами. Пусть также гессиан $\mathbf{H}^{-1}$ функции потерь $\mathcal{L}$ является обратимым в каждой стационарной точке, тогда:
\begin{equation}
\label{eq:1}
\begin{aligned}
    \nabla_{\mathbf{h}}\mathcal{Q}\left(\mathsf{T}\left(\bm{\Theta}_{0}, \mathbf{h}\right), \mathbf{h}\right) = \nabla_{\mathbf{h}}\nabla_{\bm{\Theta}}\mathcal{L}\left(\bm{\Theta}^{\eta}, \mathbf{h}\right)^{\mathsf{T}}\mathbf{H}^{-1}\nabla_{\bm{\Theta}}\mathcal{Q}\left(\bm{\Theta}^{\eta}, \mathbf{h}\right).
\end{aligned}
\end{equation}
\end{theorem}

\begin{proof}

\begin{equation}
\label{eq:2}
	\begin{aligned}
		&\nabla_{\bm{\Theta}}\mathcal{L}\left(\mathsf{T}\left(\bm{\Theta}_{0}, \mathbf{h}\right)\right) = 0 \Rightarrow\\
		\Rightarrow&\nabla_{\mathbf{h}}\left(\nabla_{\bm{\Theta}}\mathcal{L}\left(\mathsf{T}\left(\bm{\Theta}_{0}, \mathbf{h}\right)\right)\right) = \nabla_{\bm{\Theta}, \mathbf{h}}\mathcal{L}\left(\bm{\Theta}^{\eta}, \mathbf{h}\right) + \nabla_{\bm{\Theta}}^{2}\mathcal{L}\left(\bm{\Theta}^{\eta}, \mathbf{h}\right)\frac{\partial \bm{\Theta}}{\partial \mathbf{h}} = 0 \Rightarrow \\
		\Rightarrow&\frac{\partial \bm{\Theta}}{\partial \mathbf{h}} = -\left(\nabla_{\bm{\Theta}}^{2}\mathcal{L}\left(\bm{\Theta}^{\eta}, \mathbf{h}\right)\right)^{-1}\nabla_{\bm{\Theta}, \mathbf{h}}\mathcal{L}\left(\bm{\Theta}^{\eta}, \mathbf{h}\right).
 	\end{aligned}
\end{equation}

\begin{equation}
\label{eq:3}
	\begin{aligned}
		&\nabla_{\mathbf{h}}\mathcal{Q}\left(\mathsf{T}\left(\bm{\Theta}_{0}, \mathbf{h}\right)\right) = \nabla_{\mathbf{h}}\mathcal{Q}\left(\bm{\Theta}^{\eta}, \mathbf{h}\right)+\nabla_{\bm{\Theta}}\mathcal{Q}\left(\bm{\Theta}^{\eta}, \mathbf{h}\right)^{\mathsf{T}}\frac{\partial \bm{\Theta}}{\partial \mathbf{h}}.
 	\end{aligned}
\end{equation}
Подставляя~\eqref{eq:2} в~\eqref{eq:3} получаем:
\begin{equation}
\label{eq:4}
	\begin{aligned}
		\nabla_{\mathbf{h}}\mathcal{Q}\left(\mathsf{T}\left(\bm{\Theta}_{0}, \mathbf{h}\right)\right) &= \nabla_{\mathbf{h}}\mathcal{Q}\left(\bm{\Theta}^{\eta}, \mathbf{h}\right) - \nabla_{\bm{\Theta}}\mathcal{Q}\left(\bm{\Theta}^{\eta}, \mathbf{h}\right)^{\mathsf{T}}\left(\nabla_{\bm{\Theta}}^{2}\mathcal{L}\left(\bm{\Theta}^{\eta}, \mathbf{h}\right)\right)^{-1}\nabla_{\bm{\Theta}, \mathbf{h}}\mathcal{L}\left(\bm{\Theta}^{\eta}, \mathbf{h}\right) = \\
		&= \nabla_{\mathbf{h}}\mathcal{Q}\left(\bm{\Theta}^{\eta}, \mathbf{h}\right) -  \nabla_{\bm{\Theta}, \mathbf{h}}\mathcal{L}\left(\bm{\Theta}^{\eta}, \mathbf{h}\right)^{\mathsf{T}}\left(\nabla_{\bm{\Theta}}^{2}\mathcal{L}\left(\bm{\Theta}^{\eta}, \mathbf{h}\right)\right)^{-1}\nabla_{\bm{\Theta}}\mathcal{Q}\left(\bm{\Theta}^{\eta}, \mathbf{h}\right)   
 	\end{aligned}
\end{equation}
\end{proof}

\end{document} 

