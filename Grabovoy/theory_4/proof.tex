\documentclass[12pt, twoside]{article}
\usepackage[utf8]{inputenc}
\usepackage[english,russian]{babel}

\usepackage{amsthm}
\usepackage{a4wide}
\usepackage{graphicx}
\usepackage{caption}
\usepackage{amssymb}
\usepackage{amsmath}
\usepackage{mathrsfs}
\usepackage{euscript}
\usepackage{graphicx}
\usepackage{subfig}
\usepackage{caption}
\usepackage{color}
\usepackage{bm}
\usepackage{tabularx}
\usepackage{adjustbox}

\newtheorem{theorem}{Теорема}


\begin{document}
\begin{theorem}
При устремления параметра температуры~$\tau\to+\infty$ плотность \textbf{Gumbel--Softmax} концентрируется в центре симплекса.
\end{theorem}

\begin{proof}
Для доказательства рассмотрим то, как сэмплируется вектор из распределения~$\textbf{Gumbel--Softmax}\left(\textbf{s}, \tau\right)$\footnotetext[1]{https://arxiv.org/pdf/1611.01144.pdf}:
\begin{equation*}
\begin{aligned}
y_i = \frac{\exp\left(\left(s_i+g_i\right)/ \tau\right)}{\sum_{j}^{K} \exp\left(\left(s_j+g_j\right)/ \tau\right)},
\end{aligned}
\end{equation*}
где~$g_j$ сэмпл из $\textbf{Gumbel}\left(0, 1\right)$.
Тогда устремляя~$\tau\to \infty$ получаем:
\begin{equation*}
\begin{aligned}
y_i \to_{\tau\to\infty} \frac{1}{K},
\end{aligned}
\end{equation*}
откуда получаем, что все вершины распределены равномерно, следовательно плотность концентрируется в центре симплекса.
\end{proof}

\end{document} 

