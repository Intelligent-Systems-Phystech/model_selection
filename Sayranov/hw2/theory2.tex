 \documentclass[12pt]{article}
\usepackage[T2A]{fontenc}
\usepackage[utf8]{inputenc}
\usepackage[english,russian]{babel} 

\usepackage{amsmath,amsfonts,amsthm,amssymb,amsbsy,amstext,amscd,amsxtra,multicol}
\usepackage{enumitem}
\usepackage{tikz}
\usetikzlibrary{automata,positioning}
\usepackage{multicol}
\usepackage{graphicx}
\usepackage{xcolor}
\usepackage[colorlinks,urlcolor=blue]{hyperref}

\DeclareMathOperator{\Tr}{Tr}

\definecolor{linkcolor}{HTML}{799B03} % цвет ссылок
\definecolor{urlcolor}{HTML}{799B03} % цвет гиперссылок
\hypersetup{pdfstartview=FitH,  linkcolor=linkcolor,urlcolor=urlcolor, colorlinks=true}

\hoffset=0mm
\voffset=0mm
\textwidth=179mm        % ширина текста
\oddsidemargin=-5.5mm   % левое поле 25.4 - 5.4 = 20 мм
%\textheight=230mm       % высота текста 297 (A4) - 40
%\topmargin=-15.4mm      % верхнее поле (10мм)
%\headheight=5mm      % место для колонтитула
%\headsep=2mm          % отступ после колонтитула
%\footskip=7.5mm         % отступ до нижнего колонтитула

\begin{document}
\noindent\textbf{Утверждение.}
Пусть $L$ - дифференцируемая фукнция, такая что все стационарные точки $L$ являются локальными минимумами. Пусть также гессиан $H^{-1}$ функции потерь $L$ является обратимым в каждой стационарной точке. Тогда
$$\nabla_hQ(T(\theta_0, h),h)=\nabla_hQ(\theta^{\eta}, h) - \nabla_h\nabla_\theta L(\theta^\eta,h)^\top H^{-1}\nabla_\theta Q(\theta^\eta, h)$$
\noindent\textbf{Доказательство.}
Условие оптимальности:
$$\nabla_\theta L(T(\theta_0, h)) = 0$$
Отсюда получаем
$$0 = \nabla_h\left(\nabla_\theta L(T(\theta_0, h))\right) = \nabla^2_{\theta,h}L(\theta^\eta, h) + \nabla^2_\theta L(\theta^\eta, h)\theta'_h$$
Следовательно
$$\theta'_h = (\nabla^2_\theta L(\theta^\eta, h))^{-1}\nabla^2_{\theta,h}L(\theta^\eta, h)$$
Тогда полный градиент
$$\nabla_hQ(T(\theta_0, h), h) = \nabla_hQ(\theta^{\eta}, h) + \nabla_\theta Q(\theta^\eta, h)^\top\theta'_h$$
Запишется как
$$\nabla_hQ(T(\theta_0, h), h) = \nabla_hQ(\theta^{\eta}, h) + \nabla_\theta Q(\theta^\eta, h)^\top(\nabla^2_\theta L(\theta^\eta, h))^{-1}\nabla^2_{\theta,h}L(\theta^\eta, h) = $$
$$ = \nabla_hQ(\theta^{\eta}, h) + \nabla^2_{\theta,h}L(\theta^\eta, h)^\top(\nabla^2_\theta L(\theta^\eta, h))^{-1}\nabla_\theta Q(\theta^\eta, h)$$
\end{document}