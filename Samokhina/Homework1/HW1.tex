\documentclass[a4paper,12pt,notitlepage]{article}
\usepackage[T2A]{fontenc}
\def\R{{\mathbb{R}}}
\def\Q{{\cal Q}}
\def\w{{\textbf{w}}}
\def\h{{\textbf{h}}}
\def\y{{\textbf{y}}}
\def\X{{\textbf{X}}}
\usepackage[utf8]{inputenc}
\usepackage[russian]{babel}
\usepackage{amssymb,amsmath}
\usepackage{geometry}
\geometry{left=2cm}
\geometry{right=1.5cm}
\geometry{top=1.5cm}
\geometry{bottom=1.5cm}
\textheight=25cm % высота текста
\textwidth=16cm % ширина текста
\oddsidemargin=0pt % отступ от левого края
\topmargin=0pt % отступ от верхнего края
\parindent=0pt % абзацный отступ
\parskip=0pt % интервал между абзацами
\tolerance=2000 % терпимость к "жидким" строкам
\flushbottom % выравнивание высоты страниц
\usepackage{fancyhdr}
 
\pagestyle{fancy}
\fancyhf{}
\rhead{Самохина Алина}
\lhead{Задание по курсу "Выбор модели глубокого обучения"}
\rfoot{Page \thepage}


\begin{document}
%\title{Домашнее задание №1\\Выбор модели глубокого обучения}
%\author{Самохина Алина, 904a}
%\date{\today}
%\maketitle

\section*{Утверждение 1}

Максимизация вариационной нижней оценки
$$ \int_{\w}q(\w)\log\frac{p(\y, \w, | \X, \h)}{q(\w)}d\w  $$
эквивалентна минимизации расстояния Кульбака–Лейблера между распределением
$q(\w) \in \Q$ и апостериорным распределением параметров $p(\w | \y, \X, \h)$:

$$ \hat{q} = \arg \max_{q \in \Q} \int_{\w}q(\w)\log\frac{p(\y, \w, | \X, \h)}{q(\w)}d\w  \Longleftrightarrow \hat{q} = \arg \min_{q \in \Q} D_{KL}(q(\w)\|p(\w|\y, \X, \h)), $$

$$D_{KL}(q(\w)\|p(\w|\y, \X, \h)) =  \int_{\w}q(\w) \log(\frac{q(\w)}{p(\w|\y, \X, \h)})$$

\subsection*{Доказательство}
Рассмотрим выражения.


%%\begin{equation}
\begin{multline}
\arg \max_{q \in \Q} \int_{\w}q(\w)\log\frac{p(\y, \w, | \X, \h)}{q(\w)}d\w = \\
 =\arg \min_{q \in \Q} \int_{\w}-q(\w)\log\frac{p(\y, \w, | \X, \h)}{q(\w)}d\w = \\
 = \arg \min_{q \in \Q} \int_{\w}q(\w)\log\frac{q(\w)}{p(\y, \w, | \X, \h)}d\w  
\end{multline}


%%\end{equation}
\begin{equation}
\arg \min_{q \in \Q} D_{KL}(q(\w)\|p(\w|\y, \X, \h)) = \arg \min_{q \in \Q} \int_{\w}q(\w) \log\frac{q(\w)}{p(\w|\y, \X, \h)} d\w  
\end{equation}

Тогда утверждение теоремы равнозначно следующему утверждению:
\begin{equation}
  p(\y, \w, | \X, \h) \sim  p(\w|\y, \X, \h)
\end{equation}

\begin{equation}
  p(\w|\y, \X, \h) = \frac{p(\y|\w, \X, \h)p(\w|\X, \h)}{\int_{\w} p(\y|\w)p(\w|\X, \h) d\w}
\end{equation}

\begin{equation}
  p(\y, \w, | \X, \h)= {p(\y|\w, \X, \h)p(\w|\X, \h)} 
\end{equation}

Учитывая, что интеграл в знаменателе (4) из формулы Байеса ялвяется нормировочной константой, то утверждение (3) является верным, ч.т.д.
\end{document}