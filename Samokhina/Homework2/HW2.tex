\documentclass[a4paper,12pt,notitlepage]{article}
\usepackage[T2A]{fontenc}
\def\R{{\mathbb{R}}}
\def\Q{{\cal Q}}
\def\L{{\textbf{L}}}
\def\t{{\boldsymbol{\theta}}}
\def\w{{\textbf{w}}}
\def\h{{\textbf{h}}}
\def\H{{\textbf{H}}}
\def\y{{\textbf{y}}}
\def\X{{\textbf{X}}}
\usepackage[utf8]{inputenc}
\usepackage[russian]{babel}
\usepackage{amssymb,amsmath}
\usepackage{geometry}
\geometry{left=2cm}
\geometry{right=1.5cm}
\geometry{top=1.5cm}
\geometry{bottom=1.5cm}
\textheight=25cm % высота текста
\textwidth=16cm % ширина текста
\oddsidemargin=0pt % отступ от левого края
\topmargin=0pt % отступ от верхнего края
\parindent=0pt % абзацный отступ
\parskip=0pt % интервал между абзацами
\tolerance=2000 % терпимость к "жидким" строкам
\flushbottom % выравнивание высоты страниц
\usepackage{fancyhdr}
 
\pagestyle{fancy}
\fancyhf{}
\rhead{Самохина Алина}
\lhead{Задание по курсу "Выбор модели глубокого обучения"}
\rfoot{Page \thepage}


\begin{document}
%\title{Домашнее задание №1\\Выбор модели глубокого обучения}
%\author{Самохина Алина, 904a}
%\date{\today}
%\maketitle

\section*{Утверждение 1}
Пусть L — дифференцируемая функция, такая что все стационарные точки L  являются локальными минимумами. Пусть также гессиан $\H^{-1}$ функции потерь L является обратимым в каждой стационарной точке.
Тогда
$$ \nabla_{\h}\Q (T(\t_0, \h), \h) = \nabla_\h\Q(\t^{\eta}, \h) - 
\nabla_\h \nabla_\t L(\t^\eta, \h)^T\H^{-1}\nabla_\t\Q(\t^\eta, \h)$$

\subsection*{Доказательство}

Рассматриваем $\Q(T(\t_0, \h)) \Longrightarrow L(T(\t_0, \h))$.
\\

По условию утверждения $\nabla_\t L(T(\t_0, \h))=0$


$$\Longrightarrow \nabla_\h (\nabla_\t L(T(\t_0, \h))) = \nabla_\t \nabla_\h L(\t^\eta, \h)+
\nabla_\t^2 L(\t^\eta, \h)\frac{\partial\t}{\partial\h} = 0$$

$$
\nabla_\t^2 L(\t^\eta, \h)\frac{\partial\t}{\partial\h} = -\nabla_\t \nabla_\h L(\t^\eta, \h)
$$

$$
\frac{\partial\t}{\partial\h} = -(\nabla_\t^2 L(\t^\eta, \h))^{-1}\nabla_\t \nabla_\h L(\t^\eta, \h)
$$
\\

Также известно, что
$
T(\t, \h) = \t - \beta\nabla L(\t, \h)
$
\\

$$
\Longrightarrow \nabla_\h \Q(T(\t_0, \h)) = \nabla_\h \Q(\t^\eta, \h) + \nabla_\t \Q(\t^\eta, \h)^T\frac{\partial\t}{\partial\h}
$$

$$
\nabla_\h \Q(T(\t_0, \h)) = \nabla_\h \Q(\t^\eta, \h)- 
\nabla_\h \nabla_\t L(\t^\eta, \h)^T\H^{-1}\nabla_\t\Q(\t^\eta, \h)
$$

\end{document}