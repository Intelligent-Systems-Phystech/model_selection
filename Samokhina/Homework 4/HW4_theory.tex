\documentclass[a4paper,12pt,notitlepage]{article}
\usepackage[T2A]{fontenc}
\def\R{{\mathbb{R}}}
\def\Q{{\cal Q}}
\def\L{{\textbf{L}}}
\def\t{{\boldsymbol{\theta}}}
\def\gam{{\boldsymbol{\Gamma}}}
\def\lam{{\boldsymbol{\lambda}}}
\def\w{{\textbf{w}}}
\def\h{{\textbf{h}}}
\def\H{{\textbf{H}}}
\def\y{{\textbf{y}}}
\def\X{{\textbf{X}}}

\usepackage[utf8]{inputenc}
\usepackage[russian]{babel}
\usepackage{amssymb,amsmath}
\usepackage{geometry}
\geometry{left=2cm}
\geometry{right=1.5cm}
\geometry{top=1.5cm}
\geometry{bottom=1.5cm}
\textheight=25cm % высота текста
\textwidth=16cm % ширина текста
\oddsidemargin=0pt % отступ от левого края
\topmargin=0pt % отступ от верхнего края
\parindent=0pt % абзацный отступ
\parskip=0pt % интервал между абзацами
\tolerance=2000 % терпимость к "жидким" строкам
\flushbottom % выравнивание высоты страниц
\usepackage{fancyhdr}
 
\pagestyle{fancy}
\fancyhf{}
\rhead{Самохина Алина}
\lhead{Задание по курсу "Выбор модели глубокого обучения"}
\rfoot{Page \thepage}
\DeclareMathOperator*{\argmin}{\arg\min}

\begin{document}
%\title{Домашнее задание №1\\Выбор модели глубокого обучения}
%\author{Самохина Алина, 904a}
%\date{\today}
%\maketitle

\section*{Утверждение 1}
Доказать, что при устремлении параметра температуры к бесконечности, плотность
Gumbel-Softmax концентрируется в центре симплекса.
(формулы Gumbel-Softmax взяты из статьи Categorical Reparameterization with Gumbel-Softmax
E.Jang et al.)

\subsection*{Доказательство}
Рассмотрим компоненты k-мерного случайного вектора
$$
y_i = \frac{exp((\log(\pi_i) + g_i)/\tau )}{\sum_{j = 1}^kexp((\log(\pi_j ) + g_j )/\tau)}$$
в которых $g_1\dots g_k$ i.i.d. и $g_j \in Gumbel(0,1) \ \forall j=\overline{1, k}$

Устремляя $\tau\rightarrow\inf$ имеем:
$$\lim_{\tau \rightarrow \inf}y_i = \frac{1}{k}$$
$\Longrightarrow$ все компоненты рассмтариваемого вектора распределены равномерно\\ $\Longrightarrow$ плотность концентрируется в центре симплекса 
\end{document}