\documentclass[a4paper,14pt]{extarticle}
\usepackage[utf8]{inputenc}
\usepackage[english,russian]{babel}

\usepackage{amsthm}
\usepackage{graphicx}
\usepackage{caption}
\usepackage{amssymb}
\usepackage{amsmath}
\usepackage{mathrsfs}
\usepackage{euscript}
\usepackage{graphicx}
\usepackage{subfig}
\usepackage{caption}
\usepackage{color}
\usepackage{bm}
\usepackage{tabularx}
\usepackage{adjustbox}


\usepackage[toc,page]{appendix}

\usepackage{comment}
\usepackage{rotating}

\DeclareMathOperator*{\argmax}{arg\,max}
\DeclareMathOperator*{\argmin}{arg\,min}

\newtheorem{theorem}{Теорема}
\newtheorem{lemma}[theorem]{Лемма}
\newtheorem{corollary}{Следствие}[theorem]
\newtheorem{definition}{Определение}[section]
\newtheorem{example}{Пример}

\numberwithin{equation}{section}

\newcommand*{\No}{No.}
\newcommand{\overbar}[1]{\mkern 1.5mu\overline{\mkern-1.5mu#1\mkern-1.5mu}\mkern 1.5mu}
\begin{document}
\section{Утверждение}
\begin{theorem}
	Пусть $h \in \mathcal{F}^1$ --- дифференцируемая выпуклая функция. Пусть также гессиан $\nabla^2_1 h$ функции потерь обратимым на множестве оптимальных точек, тогда:
$$
	\nabla_\lambda g\left(\mathsf{T}\left(\bm{X}_{0}, \lambda\right), \lambda\right) = \nabla_{\lambda} g(X^{\nu}(\lambda),\lambda)-\left(\nabla_{X,\lambda}^{2} h\right)^{T}\left(\nabla_{X}^{2} h\right)^{-1} \nabla_{X} g(X^{\nu}(\lambda),\lambda)
$$
\end{theorem}
\begin{proof}
	Запишем условие оптимальности первого порядка
	$$\nabla_{X} h(X(\lambda), \lambda)=0$$
	продифференцируем это уравнение
	$$
	\nabla_{X,\lambda}^{2} h+\nabla_{X}^{2} h \cdot \mathrm{D} X=0
	$$
	выразим производную $X$, используя обратимость гессиана.
	$$
	\mathrm{D} X = - \left(\nabla_{X}^{2} h\right)^{-1}\nabla_{X,\lambda}^{2} h
	$$
	Теперь запишем полный градиент
	$$
		\nabla_\lambda g\left(\mathsf{T}\left(\bm{X}_{0}, \lambda\right), \lambda\right) = \nabla_{\lambda} g(X^{\nu}(\lambda),\lambda) +  \nabla_{X} g(X^{\nu}(\lambda),\lambda) \mathrm{D} X
	$$
	Используем знание от том, что $A^Tb = b^TA$ и подставляем выражение для $\left(\mathrm{D} X\right)^T = - \left(\left(\nabla_{X}^{2} h\right)^{-1}\nabla_{X,\lambda}^{2} h\right)^T = \left(\nabla_{X,\lambda}^{2} h\right)^T\left(\nabla_{X}^{2} h\right)^{-T}$, поскольку гессиан симметричен, то $-T$ переходит в $-1$. Подставляем
	$$
	\nabla_\lambda g\left(\mathsf{T}\left(\bm{X}_{0}, \lambda\right), \lambda\right) = \nabla_{\lambda} g(X^{\nu}(\lambda),\lambda)-\left(\nabla_{X,\lambda}^{2} h\right)^{T}\left(\nabla_{X}^{2} h\right)^{-1} \nabla_{X} g(X^{\nu}(\lambda),\lambda)
	$$
	Что и требовалось доказать.
\end{proof}
\end{document}